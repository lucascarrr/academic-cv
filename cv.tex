% tectonic cv.tex

%%%%%%%%%%%%%%%%%%%%%%%%%%%%%%%%%%%%%%%%%%%%%%%%%%%%%%%%%%%%%%%%%%%%%%%%%%%%%%%
% A clean template for an academic CV. This is a short summary version.
%
% Uses tabularx to create two column entries (date and job/edu/citation).
% Defines commands to make adding entries simpler.
%
%%%%%%%%%%%%%%%%%%%%%%%%%%%%%%%%%%%%%%%%%%%%%%%%%%%%%%%%%%%%%%%%%%%%%%%%%%%%%%%

\documentclass[10pt,a4paper]{article}

% Identifying information
\newcommand{\Title}{Academic CV}
\newcommand{\FirstName}{Bartosz}
\newcommand{\LastName}{Bednarczyk}
\newcommand{\Initials}{BBe}
\newcommand{\MyName}{\FirstName\ \LastName}
\newcommand{\Me}{\underline{\LastName, \Initials}}  % For citations
\newcommand{\Email}{bartosz.bednarczyk@cs.uni.wroc.pl}
\newcommand{\PersonalWebsite}{bartoszjanbednarczyk.github.io}
\newcommand{\LabWebsite}{bartoszjanbednarczyk.github.io}
\newcommand{\ORCID}{0000-0002-8267-7554}
\newcommand{\GitHubProfile}{bartoszjanbednarczyk}


% Load packages
%%%%%%%%%%%%%%%%%%%%%%%%%%%%%%%%%%%%%%%%%%%%%%%%%%%%%%%%%%%%%%%%%%%%%%%%%%%%%%%

% Full Unicode support for non-ASCII characters
\usepackage[utf8]{inputenc}
\usepackage[english]{babel}
\usepackage[TU]{fontenc}

% Set main fonts
\usepackage[sfdefault]{atkinson}
\usepackage[ttdefault]{sourcecodepro}

% Icon fonts
\usepackage{fontawesome5}
\usepackage{academicons}

% Disable hyphenation
\usepackage[none]{hyphenat}

% Control the font size
\usepackage{anyfontsize}

% For fancy and multipage tables
\usepackage{tabularx}
\usepackage{ltablex}

% For new environments
\usepackage{environ}

% Manage dates and times
\usepackage{datetime}

% Set the page margins
\usepackage{geometry}

% To get the total page numbers (\pageref{LastPage})
\usepackage{lastpage}

% Control spacing in enumerates
\usepackage{enumitem}

% Use custom colors
\usepackage[usenames,dvipsnames]{xcolor}

% Configure section titles
\usepackage{titlesec}

% Fancy header configuration
\usepackage{fancyhdr}

% Control PDF metadata and links
\usepackage[colorlinks=true]{hyperref}


% Template configuration
%%%%%%%%%%%%%%%%%%%%%%%%%%%%%%%%%%%%%%%%%%%%%%%%%%%%%%%%%%%%%%%%%%%%%%%%%%%%%%%

\geometry{%
  margin=12.5mm,
  headsep=0mm,
  headheight=0mm,
  footskip=5mm,
  includehead=true,
  includefoot=true
}

% Custom colors
\definecolor{mediumgray}{gray}{0.5}
\definecolor{lightgray}{gray}{0.9}
\definecolor{mediumblue}{HTML}{2060c2}
\definecolor{lightblue}{HTML}{a0c3ff}

% No indentation
\setlength\parindent{0cm}

% Increase the line spacing
\renewcommand{\baselinestretch}{1.1}
% and the spacing between rows in tables
\renewcommand{\arraystretch}{1.25}

% Remove space between items in itemize and enumerate
\setlist{nosep}

% Set the spacing and format of sections
\titleformat{\section}
  {\normalfont\Large\mdseries} % format
  {} % label
  {0pt} % separation (left separation for hang)
  {} % text before title
  [\titlerule] % text after title
\titlespacing*{\section}
  {0pt} % left pad
  {0.1cm} % before
  {0cm} % after

% Disable number of sections. Use this instead of "section*" so that the sections still
% appear as PDF bookmarks. Otherwise, would have to add the table of contents entries
% manually.
\makeatletter
\renewcommand{\@seccntformat}[1]{}
\makeatother

% Define a new environment to place all CV entries in a 2-column table.
% Left column are the dates, right column the entries.
\newcommand{\TablePad}{\vspace{-0.2cm}}
\NewEnviron{EntriesTableDuration}{
\TablePad
\begin{tabularx}{\textwidth}{@{}p{0.135\textwidth}@{\hspace{0.02\textwidth}}p{0.845\textwidth}@{}}
  \BODY
\end{tabularx}
\TablePad
}
\NewEnviron{EntriesTableYear}{
\TablePad
\begin{tabularx}{\textwidth}{@{}p{0.08\textwidth}@{\hspace{0.01\textwidth}}p{0.91\textwidth}@{}}
  \BODY
\end{tabularx}
\TablePad
}

% Macros to set the year and duration on the left column
\newcommand{\Duration}[2]{\fontsize{10pt}{0}\selectfont \texttt{#1-#2}}
\newcommand{\Year}[1]{\fontsize{10pt}{0}\selectfont \texttt{#1}}
\newcommand{\Ongoing}{on}
\newcommand{\Future}{future}

% Macros to add links and mark publications
\newcommand{\DOI}[1]{DOI: \href{https://doi.org/#1}{#1}}
\newcommand{\Website}[1]{\href{https://#1}{#1}}
\newcommand{\Preprint}[1]{Preprint: \href{https://doi.org/#1}{#1}}
\newcommand{\GitHub}[1]{GitHub: \href{https://github.com/#1}{#1}}

% Define command to insert month name and year as date
\newdateformat{monthyear}{\monthname[\THEMONTH], \THEYEAR}

% Configure a fancy footer
\newcommand{\Separator}{\hspace{3pt}|\hspace{3pt}}
\newcommand{\FooterFont}{\footnotesize\color{mediumgray}}
\pagestyle{fancy}
\fancyhf{}
\lfoot{%
  \FooterFont{}
  \MyName{}
  \Separator{}
  \Title{}
}
\rfoot{%
  \FooterFont{}
  Last updated: \monthyear\today{}
  \Separator{}
  \thepage\space of\space \pageref*{LastPage}
}
\renewcommand{\headrulewidth}{0pt}
\renewcommand{\footrulewidth}{1pt}
\preto{\footrule}{\color{lightgray}}

% Metadata for the PDF output and control of hyperlinks
\hypersetup{
  colorlinks,
  allcolors=mediumblue,
  breaklinks=true,
  pdftitle={\Title{} - \MyName},
  pdfauthor={\MyName},
}


%%%%%%%%%%%%%%%%%%%%%%%%%%%%%%%%%%%%%%%%%%%%%%%%%%%%%%%%%%%%%%%%%%%%%%%%%%%%%%%
\begin{document}

\begin{minipage}[t]{0.5\textwidth}
  {\fontsize{20pt}{0}\selectfont\MyName}
\end{minipage}
\begin{minipage}[t]{0.5\textwidth}
  \begin{flushright}
    \Title{}
  \end{flushright}
\end{minipage}
\\[-0.1cm]
\textcolor{lightgray}{\rule{\textwidth}{3pt}}
\begin{minipage}[t]{0.5\textwidth}
  ORCID: \href{https://orcid.org/\ORCID}{\ORCID}
  \\
  Website: \Website{\PersonalWebsite}
  \\
  Email: \href{mailto:\Email}{\Email}
\end{minipage}
\begin{minipage}[t]{0.5\textwidth}
  \begin{flushright}
  Institute of Computer Science, University of Wrocław 
  \\
  Fryderyka Joliot-Curie 15, 50-383 Wrocław, Poland
  \\
  Computational Logic Group, ICCL@TU Dresden,
  \\
  Nöthnitzer Str. 46, 01187 Dresden, Germany 
  \end{flushright}
\end{minipage}
\vspace{0.3cm}

% %%%%%%%%%%%%%%%%%%%%%%%%%%%%%%%%%%%%%%%%%%%%%%%%%%%%%%%%%%%%%%%%%%%%%%%%%%%%%%%

\section{Education}

\begin{EntriesTableDuration}
  \Year{$\sim$04.2024}  &
  \textbf{PhD in Computer Science}, Technische Universität Dresden, Germany. \newline
  Thesis: \emph{Database-Inspired Reasoning Problems in Description Logics With Path Expressions} \newline
  Supervisors: Sebastian Rudolph (TU Dresden) and Emanuel Kieroński (University of Wrocław) \newline
  Reviewers: Sebastian Rudolph (TU Dresden) and Magdalena Ortiz (TU Wien)
  \\
  \Year{01.10.2018} & \textbf{M1 Master Parisien de Recherche en Informatique}, École Normale Supérieure Paris-Saclay, France \newline
  Thesis: \emph{Assertion Languages with Modalities and Separating Connectives} \newline
  Supervisor: Stéphane Demri (LMF, CNRS \& ENS Paris-Saclay) \newline
  Grade: Magna cum laude
  \\
  \Year{15.02.2017} & \textbf{BSc in Computer Science}, University of Wrocław, Poland \newline
  Thesis: \emph{Satisfiability of the Two-Variable Fragment of FO with Counting Quantifiers over Finite Trees} \newline
  Supervisor: Witold Charatonik (University of Wrocław) \newline
\end{EntriesTableDuration}

%%%%%%%%%%%%%%%%%%%%%%%%%%%%%%%%%%%%%%%%%%%%%%%%%%%%%%%%%%%%%%%%%%%%%%%%%%%%%%%

\section{Positions}

\begin{EntriesTableDuration}
  \Duration{04.2019}{Now}  &
  \textbf{Research Associate}, Technische Universität Dresden, Germany
\end{EntriesTableDuration}

%%%%%%%%%%%%%%%%%%%%%%%%%%%%%%%%%%%%%%%%%%%%%%%%%%%%%%%%%%%%%%%%%%%%%%%%%%%%%%%

\section{Participation in Research Grants}

\begin{EntriesTableDuration}
  \Duration{04.'19}{09.'23}  & 
  \textbf{Research Associate}: ERC Consolidator Grant DeciGUT (PI: Sebastian Rudolph)\newline
  \emph{Hosting institution}: TU Dresden, Germany
\end{EntriesTableDuration}

\begin{EntriesTableDuration}
  \Duration{10.'18}{09.'22}  & 
  \textbf{Principal Investigator}: Polish Min. of Science and H. Education ``Diamond Grant''  DI2017006447.\newline
  \emph{Hosting institution}: University of Wrocław, Poland
\end{EntriesTableDuration}

\begin{EntriesTableDuration}
  \Duration{03.'17}{06.'18}  & 
  \textbf{Student Assistant}: Polish National Science Centre Grant 2016/21/B/ST6/01444 (PI: E. Kieroński)\newline
  \emph{Hosting institution}: University of Wrocław, Poland
\end{EntriesTableDuration}

%%%%%%%%%%%%%%%%%%%%%%%%%%%%%%%%%%%%%%%%%%%%%%%%%%%%%%%%%%%%%%%%%%%%%%%%%%%%%%%

\section{Longer Research Stays}

\begin{EntriesTableYear}
  \Year{2021 (2mth)}  & 
  Research visit (Topic: Model Theory of Ordered Logics)\newline
  \emph{Hosted by}: Tampere University (Antti Kuusisto)
\end{EntriesTableYear}

\begin{EntriesTableYear}
  \Year{2018 (3mth)}  & 
  Research visits (Topic: Query Entailment in Description Logics)\newline
  \emph{Hosted by}: TU Dresden (Sebastian Rudolph)
\end{EntriesTableYear}

\begin{EntriesTableYear}
  \Year{2018 (3mth)}  & 
  Research Internship (Topic: Algebraic Characterisations of Tree Languages)\newline
  \emph{Hosted by}: University of Oxford (James Worrell \& Michaël Cadilhac)
\end{EntriesTableYear}

\begin{EntriesTableYear}
  \Year{2017 (3mth)}  & 
  Research Internship (Topic: Temporal Logics and Weighted Automata)\newline
  \emph{Hosted by}: IST Austria (Krishnendu Chatterjee)
\end{EntriesTableYear}

%%%%%%%%%%%%%%%%%%%%%%%%%%%%%%%%%%%%%%%%%%%%%%%%%%%%%%%%%%%%%%%%%%%%%%%%%%%%%%%

\section{Teaching}

\begin{EntriesTableYear}
  \Duration{'18}{$\ldots$} &
  Logic in Computer Science @ University of Wrocław (TA) \\

  \Duration{'21}{$\ldots$} &
  Computer Science Teacher @ XIV Highschool of Wrocław \\

  \Duration{'19}{'20} &
  Research Seminar in Logic and Database @ University of Wrocław \\

  \Duration{'19}{'22} &
  Finite and Algorithmic Model Theory (Lecturer + TA) \newline
  @ University of Wrocław '19 '22, @ TU Dresden '21 '22.\\

  \Duration{'19}{'29} &
  Databases @ University of Wrocław (TA) \\

\end{EntriesTableYear}

% %%%%%%%%%%%%%%%%%%%%%%%%%%%%%%%%%%%%%%%%%%%%%%%%%%%%%%%%%%%%%%%%%%%%%%%%%%%%%%%

\section{Awards}

\begin{EntriesTableDuration}
  \Year{2023}  &
  Best student paper at JELIA 2023\\
\end{EntriesTableDuration}

\begin{EntriesTableDuration}
  \Year{2021}  &
  Award for Outstanding Young Scientists\newline Give by the Polish Ministry of Science and Higher Education (194 040 PLN)\\
\end{EntriesTableDuration}

\begin{EntriesTableDuration}
  \Year{2021}  &
  Best student paper at JELIA 2021\\
\end{EntriesTableDuration}

\begin{EntriesTableDuration}
  \Year{2019}  &
  Hugo Steinhaus Scholarship for Mathematical Sciences\newline Award for outstanding Phd students from Wrocław (18.000 PLN)\\
\end{EntriesTableDuration}

\begin{EntriesTableDuration}
  \Year{2018}  &
  Diamond Grant by the Polish Ministry of Science and Higher Education\newline Prestigious funding for own four-year research project\\
\end{EntriesTableDuration}

%%%%%%%%%%%%%%%%%%%%%%%%%%%%%%%%%%%%%%%%%%%%%%%%%%%%%%%%%%%%%%%%%%%%%%%%%%%%%%%

\section{Professional service}

\begin{EntriesTableDuration}

  \Year{2024}  &
  Reviewer: PODS 2024 \newline
  PC: AAAI 2024\\

  \Year{2023}  &
  Reviewer: EUMAS 2023 \newline
  PC: AAAI 2023, KR 2023, JELIA 2023, IJCAI 2023, DL 2023, DPFO Workshop 2023, TIME 2023\\

  \Year{2022}  &
  Reviewer: CSL 2022, LMCS, IPL, JAIR, ToCL \newline
  PC: IJCAI 2022, KR 2022, DL 2022\\

  \Year{2021}  &
  Reviewer: DL 2021, Elsevier Artificial Intelligence, Fundamenta Informaticae \newline
  Senior PC: IJCAI 2021\\

  \Year{2020}  &
  Reviewer: KR 2020, CONCUR 2020, DL 2020\\  
\end{EntriesTableDuration}


% %%%%%%%%%%%%%%%%%%%%%%%%%%%%%%%%%%%%%%%%%%%%%%%%%%%%%%%%%%%%%%%%%%%%%%%%%%%%%%%

\section{Student Supervision}

\begin{EntriesTableDuration}

  \Year{2023}  &
  Karol Ochman-Milarski (BSc)\newline
  Title: Resolution for Forward Guarded Fragment\\

  \Year{2023}  &
  Julita Osman, Aleksandra Stępniewska, Nikola Wrona (BEng) \newline
  Title: E-learning platform supporting studying for the Matura exam in Computer Science\\

  \Year{2022}  &
  Mateusz Urbańczyk (MSc) \newline
  Title: Categorical semantics for model comparison games for description logics\\

  \Year{2022}  &
  Johannes Tantow (Großer Beleg) \newline
  Title: Interpolation for the Two-Variable Guarded Fragment with Counting\\

  \Year{2021}  &
  Oskar Fiuk (BSc) \newline
  Title: Presburger Tree Automata With Applications to Logics With Expressive Counting\\

  \Year{2021}  &
  Sebastian Zięciak (BSc) \newline
  Title: On Several Equivalent Characterisations of the Variety R\\

  \Year{2021}  &
  Maja Orłowska and Anna Pacanowska  (BSc) \newline
  Title: Statistical constructions in decidable fragments of First-Order Logic\\

  \Year{2021}  &
  Martyna Siejba (MSc) \newline
  Title: The complexity of the satisfiability problem for Modal Logics with Data over Heaps\\
    
\end{EntriesTableDuration}

% %%%%%%%%%%%%%%%%%%%%%%%%%%%%%%%%%%%%%%%%%%%%%%%%%%%%%%%%%%%%%%%%%%%%%%%%%%%%%%%

% %%%%%%%%%%%%%%%%%%%%%%%%%%%%%%%%%%%%%%%%%%%%%%%%%%%%%%%%%%%%%%%%%%%%%%%%%%%%%%%


% \begin{EntriesTableDuration}
%   \Duration{10.'18}{09.'22}  & 
%   \textbf{Principal Investigator}: Polish Min. of Science and H. Education ``Diamond Grant''  DI2017006447.\newline
%   \emph{Hosting institution}: University of Wrocław, Poland
% \end{EntriesTableDuration}

% %%%%%%%%%%%%%%%%%%%%%%%%%%%%%%%%%%%%%%%%%%%%%%%%%%%%%%%%%%%%%%%%%%%%%%%%%%%%%%%
% \section{Open Research Software}

% \begin{EntriesTableDuration}
%   \Duration{2010}{\Ongoing} &
%   \textbf{Fatiando a Terra} | \Website{www.fatiando.org}
%   \newline
%   \textit{Python tools for geophysical data processing, forward modeling, and inversion}
%   \newline
%   Role: Project founder, core developer, Steering Council Member
%   \\
%   \Duration{2017}{\Ongoing} &
%   \textbf{PyGMT} | \Website{www.pygmt.org}
%   \newline
%   \textit{A Python interface for the Generic Mapping Tools}
%   \newline
%   Role: Project founder, developer, advisor
%   \\
%   \Duration{2017}{\Ongoing} &
%   \textbf{The Generic Mapping Tools (GMT)} | \Website{www.generic-mapping-tools.org}
%   \newline
%   \textit{A data processing and mapping toolbox for the Earth, Ocean, and Planetary Science}
%   \newline
%   Role: Community stewardship advisor, set up the website + forum + GitHub workflow
%   \\
%   \Duration{2022}{\Ongoing} &
%   \textbf{xlandsat} | \Website{compgeolab.org/xlandsat}
%   \newline
%   \textit{Load Landsat remote sensing scenes in Python and xarray}
%   \newline
%   Role: Creator and sole developer
%   \\
%   \Duration{2009}{2016} &
%   \textbf{Tesseroids} | \Website{tesseroids.leouieda.com}
%   \newline
%   \textit{Forward modeling of gravitational fields in spherical coordinates}
%   \newline
%   Role: Creator and sole developer
% \end{EntriesTableDuration}


% %%%%%%%%%%%%%%%%%%%%%%%%%%%%%%%%%%%%%%%%%%%%%%%%%%%%%%%%%%%%%%%%%%%%%%%%%%%%%%%
% \section{Grants and Fellowships}

% \begin{EntriesTableDuration}
%   \Duration{2022}{2024}  &
%   \textbf{Towards individual-grain paleomagnetism: Translating regional-scale geophysics to the nascent field of magnetic microscopy}.
%   \newline
%   Royal Society.
%   \Me{} (PI); \Ricardo{}.
%   Award: \href{https://www.compgeolab.org/news/rsoc-mag-microscopy-2022.html}{IES\textbackslash{}R3\textbackslash{}213141}
%   \\
%   \Year{2020}  &
%   \textbf{SSI Fellowship Programme}.
%   \newline
%   Software Sustainability Institute.
%   \Me{} (PI).
%   Award: \Website{software.ac.uk/about/fellows}
%   \\
%   \Duration{2020}{2024}  &
%   \textbf{A Sustainable Plan for the Future of the Generic Mapping Tools}.
%   \newline
%   NSF-EAR.
%   \Paul{} (PI); \Me{}.
%   Award: \href{https://www.nsf.gov/awardsearch/showAward?AWD_ID=1948602}{1948602}.
%   \\
%   \Duration{2018}{2024}  &
%   \textbf{The EarthScope/GMT Analysis and Visualization Toolbox}.
%   \newline
%   NSF-EAR.
%   \Paul{} (PI); \Me{}; \Bridget{}.
%   Award: \href{https://www.nsf.gov/awardsearch/showAward?AWD_ID=1829371}{1829371}.
% \end{EntriesTableDuration}

% %%%%%%%%%%%%%%%%%%%%%%%%%%%%%%%%%%%%%%%%%%%%%%%%%%%%%%%%%%%%%%%%%%%%%%%%%%%%%%%
% \section{Selected Invited Presentations}

% \begin{EntriesTableYear}
% \Year{2021}  &
%   \textbf{Design useful tools that do one thing well and work together: rediscovering the UNIX philosophy while building the Fatiando a Terra project}.
%   \newline
%   AGU 2021.
%   \Me; \LLi; \Santiago; \Agustina.
%   \GitHub{fatiando/agu2021}.
%   \\
%   &
%   \textbf{Open-science for gravimetry: tools, challenges, and opportunities}.
%   \newline
%   GFZ Helmholtz Centre Potsdam.
%   \Me; \Santiago; \Agustina.
%   \GitHub{leouieda/2021-06-22-gfz}.
%   \\
% \Year{2021}  &
%   \textbf{Fatiando a Terra: Open-source tools for geophysics}.
%   \newline
%   Geophysical Society of Houston.
%   \Me; \Santiago; \Agustina.
%   \GitHub{fatiando/2021-gsh}.
%   \\
% \Year{2020}  &
%   \textbf{Geophysical research powered by open-source}.
%   \newline
%   Christian Albrechts Universität zu Kiel.
%   \Me.
%   \GitHub{leouieda/2020-07-01-kiel}.
% \end{EntriesTableYear}

% %%%%%%%%%%%%%%%%%%%%%%%%%%%%%%%%%%%%%%%%%%%%%%%%%%%%%%%%%%%%%%%%%%%%%%%%%%%%%%%
% \section{Publication Highlights}

% \begin{EntriesTableYear}
% \Year{2023}  &
%   \textbf{Full vector inversion of magnetic microscopy images using Euler deconvolution as a priori information}.
%   \newline
%   EarthArXiv.
%   \DOI{10.31223/X5QD5Z}.
%   \newline
%   \Gelson; \Me; \Ricardo; \Janine; \Roger.
%   \GitHub{compgeolab/micromag-euler-dipole}.
%   \\
% \Year{2021}  &
%   \textbf{Gradient-boosted equivalent sources}.
%   \newline
%   Geophysical Journal International.
%   \DOI{10.1093/gji/ggab297}.
%   \Preprint{10.31223/X58G7C}.
%   \newline
%   \Santiago; \Me.
%   \GitHub{compgeolab/eql-gradient-boosted}.
%   \\
% \Year{2020}  &
%   \textbf{Pooch: A friend to fetch your data files}.
%   \newline
%   Journal of Open Source Software.
%   \DOI{10.21105/joss.01943}.
%   \newline
%   \Me; \Santiago; \Remi; \Hugo; \MattTurk; \emph{et al}.
%   \GitHub{fatiando/pooch}.
%   \\
% \Year{2019}  &
%   \textbf{The Generic Mapping Tools, Version 6}.
%   \newline
%   Geochemistry, Geophysics, Geosystems.
%   \DOI{10.1029/2019GC008515}.
%   \newline
%   \Paul; \Joaquim; \Me; \Remko; \Florian; \Walter; \Dongdong.
%   \\
% \Year{2019}  &
%   \textbf{Gravitational field calculation in spherical coordinates using variable densities in depth}.
%   \newline
%   Geophysical Journal International.
%   \DOI{10.1093/gji/ggz277}.
%   \Preprint{10.31223/osf.io/3548g}.
%   \newline
%   \Santiago; \Agustina; \Gimenez; \Me.
%   \GitHub{pinga-lab/tesseroid-variable-density}.
%   \\
% \Year{2019}  &
%   \textbf{Efficient 3D large-scale forward-modeling and inversion of gravitational fields in spherical coordinates with application to lunar mascons}.
%   \newline
%   Journal of Geophysical Research: Solid Earth.
%   \DOI{10.1029/2019JB017691}.
%   \Preprint{10.31223/osf.io/dzf9j}.
%   \newline
%   \Guangdong; \Bo; \Me; \JLiu; \MKaban; \LChen; \RGuo.
%   \\
% \Year{2018}  &
%   \textbf{Verde: Processing and gridding spatial data using Green's functions}.
%   \newline
%   Journal of Open Source Software.
%   \DOI{10.21105/joss.00957}.
%   \newline
%   \Me.
%   \GitHub{fatiando/verde}.
%   \\
% \Year{2017}  &
%   \textbf{Fast non-linear gravity inversion in spherical coordinates with application to the South American Moho}.
%   \newline
%   Geophysical Journal International.
%   \DOI{10.1093/gji/ggw390}.
%   \Preprint{10.31223/osf.io/9ba4m}.
%   \newline
%   \Me; \Val.
%   \GitHub{pinga-lab/paper-moho-inversion-tesseroids}.
%   \\
% \Year{2016}  &
%   \textbf{Tesseroids: forward modeling gravitational fields in spherical coordinates}.
%   \newline
%   Geophysics.
%   \DOI{10.1190/geo2015-0204.1}.
%   \newline
%   \Me; \Val; \Carla.
%   \GitHub{pinga-lab/paper-tesseroids}.
% \end{EntriesTableYear}

\end{document}
